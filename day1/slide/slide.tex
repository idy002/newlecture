\documentclass[11pt,serif]{beamer}
\usepackage[utf8]{inputenc}
\usepackage{xeCJK}
\usepackage[T1]{fontenc}
\usepackage{amsmath}
\usepackage{amsfonts}
\usepackage{amssymb}
\usepackage{bbm}
\usepackage{graphicx}
\usetheme{Boadilla}

\begin{document}
	\author{丁尧尧}
	\title{题目选讲}
	\date{\today}
	
	\begin{frame}[plain]
	\maketitle
	\end{frame} 

	\begin{frame}{百度之星2018复赛 T1}
		
		\begin{problem}{没有兄弟的舞会}
			给你一棵$n$个点有根树,每个点有个权值,如果两个点有相同的父亲,则他们是兄弟。你现在需要找一个点集,其中最多有一对点是兄弟关系,问这个点集的权值和是多少?
			
			$1 \leq n \leq 10^5$.
		\end{problem}
		\pause
		\begin{solution}
			贪心或树型DP
		\end{solution}
	\end{frame}

	\begin{frame}{百度之星2018复赛 T2}
		\begin{problem}[序列期望]
			令$X_1,X_2,\dots,X_n$是整数随机变量,其中$X_i$是从$[l_i,r_i]$中随机选择的一个整数,令
			\begin{align}
				h & = max(X_1,X_2,\dots,X_n) \\
				Y & = \prod_{i = 1}^{n}(h + 1 - X_i)
			\end{align}
			问模$10^9+7$意义下$\mathbb{E}[Y]$.
			
			$1 \leq n \leq 100, 1 \leq l_i \leq r_i \leq 10^4$.
		\end{problem}
		\pause
		\begin{solution}
			枚举$h$
		\end{solution}
	\end{frame}

	\begin{frame}{百度之星2018复赛 T3}
		\begin{problem}[带劲的and和]
			给你一个$n$个点$m$条边的无向图,每个点有一个非负权值,点$i$的权值用$v_i$表示,令
			$$f(i,j) = \mathbbm{1}_{\text{$i$和$j$连通}}$$求:
			$$
				\sum_{i=1}^{n-1} \sum_{j=i+1}^{n} f(i,j) \times \max(v_i, v_j)\times (v_i \& v_j)
			$$
			
			$1 \leq n \leq 10^5$
		\end{problem}
		
		\pause 
	
		\begin{solution}
			先求连通块,对于同一个连通块的所有权值从小到达排序,然后从前往后算,其中$v_i \& v_j$需要记录每一位出现多少次。
		\end{solution}
	\end{frame}

	\begin{frame}{2018 Multi-University Training Contest 8}
		
		\begin{problem}[Character Encoding]
			求
			$$
				\sum_{i = 1}^{m}x_i = k \quad (0 \leq x_i < n)
			$$
			的方案数(模$10^9+7$)。
			
			$1 \leq m, k, n \leq 10^5$.
		\end{problem}
	
		\pause
		
		\begin{solution}
			容斥
		\end{solution}
	
	\end{frame}

	\begin{frame}{2018 Multi-University Training Contest 8}
		\begin{problem}[Card Game]
			给定$n$张卡片,每张卡片正反面各有一个数。问至少要翻转多少张卡片,才能使正面向上的数互不相同,并求方案数。
		\end{problem}
		
		\pause 
	
		\begin{solution}
			首先建图:每个数字为一个节点,每张卡片反面数字向正面数字连一条有向边。问题转化为:至少要反转多少条边的方向,才能使得每个点的入度不会超过1。我们对每个弱连通分量分别处理。易知,当底图是树或基环树时,才可能有解。对于基环树,先把环找出来,然后将环上的边的方向统一一下;非环边的方向则是唯一确定的,从环上的点向外做一遍dfs 即可。对于树,可以正反两次dfs处理出每个点作为根时所需要的反向次数,并统计出最小值以及方案数。最后将答案合并即可。
		\end{solution}
	\end{frame} 

	\begin{frame}{2018 Multi-University Training Contest 8} 
		\begin{problem}[Taotao Picks Apples]
			对于一个序列,从前往后看,每次当手上没有数或者手上的数小于当前的数,就把手上的数替换成当前的数,定义手上出现的数的个数为这个序列的可见度。给定一个长为$n$的序列,有$m$个询问,每个询问两个数$(p,q)$,表示如果把$p$位置的数换成$q$,这个序列的可见度为多少?
			
			$1 \leq n, m \leq 10^5$
		\end{problem}
		\pause
		\begin{solution}
			法一:线段树维护
			
			\pause 
			
			法二:考虑每次修改不叠加,因此我们可以从如何对原序列进行预处理着手。
			通过观察可以发现,将原序列从任意位置断开,我们可以通过分别维护左右段的某些信息来拼接得到答案。	
			对于每次询问:
			考虑这个数左边的部分加上这个数之后的答案和最大值;
			再找到右边第一个大于左半部分最大值的数,答案相加即可。
		\end{solution}
	\end{frame} 

\end{document}