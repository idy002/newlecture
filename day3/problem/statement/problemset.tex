\documentclass[11pt,a4paper,oneside]{article}
\usepackage[english]{babel}
\usepackage{olymp}
\usepackage[dvips]{graphicx}
\usepackage{color}
\usepackage{colortbl}
%\usepackage{expdlist}
%\usepackage{mfpic}
%\usepackage{comment}
\usepackage{multirow}

\usepackage{xeCJK}

%\setCJKmainfont[BoldFont={Hei}]
%{SimSun}
%\setCJKmonofont{FangSong}

\renewcommand{\contestname}{
No.7 High School OI Training \\
idy002, \today
}    

\begin{document}

\begin{problem}{keys}{keys.in}{keys.out}{1 second}{256MB}

    现在有$n$个人,$k$把钥匙,还有一个办公室,他们在$x$轴上.
    
    其中,第$i$个人的位置用一个整数$a_i$表示,第$i$把钥匙的位置用一个整数$b_i$表示,办公室的位置用整数$p$表示. 保证钥匙的数量大于等于人的数量.
    
    现在,他们每个人都需要先去拿到一把钥匙(每个人必须拿不同的钥匙),然后前往办公室. 他们每走单位长度花费一秒钟,假设他们同时出发,请问他们最少需要多少时间,可以所有人都到达办公室?

    \InputFile

	第$1$行包含$3$个整数$n$, $k$ 和 $p$;
	
	接下来$1$行包含$n$个整数$a_1,a_2,\dots,a_n$;
	
	接下来$1$行包含$k$个整数$b_1,b_2,\dots,b_n$;

    \OutputFile

	输出距离和的最小值.

    \Example

    \begin{example}
        \exmp{
            2 4 50
            20 100
            60 10 40 80
        }{
            50
        }%
    \end{example}

	样例解释:位置为$20$的人应该去拿位置为$40$的钥匙,然后前往办公室$50$,走的距离是$20 + 10 = 30$, 位置为$100$的人应该去位置位$80$的钥匙然后前往办公室$50$,走的距离是$50$, 故花费$max(30,50) = 50$秒就可以了.
	
	

    \begin{example}
		\exmp{
			1 2 10
			11
			15 7
		}{
			7
		}%
	\end{example}

	样例解释:选择位置为$7$的钥匙最优秀,时间为:$(11-7) + (10-7) = 7$.

    \Note
    
    \begin{itemize}
		\item 对于$30\%$的数据,$1 \leq n \leq 100$, $n \leq k \leq 200$;
		\item 对于$100\%$的数据,$1 \leq n \leq 5000$, $n \leq k \leq 10000$, $1 \leq a_i, b_i, p \leq 10^9$.
    \end{itemize}

\end{problem}
\begin{problem}{cards}{cards.in}{cards.out}{1 second}{256 MB}
	
	给你$n$张卡牌,每张卡牌上有一个整数,整数可以相同.
	
	现在把这$n$张卡牌按顺序叠成一叠,每次,我们将最上面的一张拿出来查看,如果它上面的数是当前剩余卡牌的最小值,则把它拿出来,否则将它放到卡堆的最后.
	
	问我们需要查看多少次,才能将所有卡牌都拿完.
	
	\InputFile
	
	第$1$行包含一个整数$n$表示桌上的卡牌数;
	
	下一行包含$n$个数$a_1,a_2,\dots,a_n$表示卡牌堆从上到下的数字.
	
	\OutputFile
	
	输出一个整数表示需要查看多少次牌.
	
	\Example
	
	\begin{example}
		\exmp{
			4
			6 3 1 2
		}{
			7
		}%
	\end{example}
	
	样例说明: 依次查看:6, 3, 1(拿出), 2(拿出), 6, 3(拿出), 6(拿出), 故需要查看7次.
	
	\begin{example}
		\exmp{
			1
			1000
		}{
			1
		}%
	\end{example}
	
	\begin{example}
		\exmp{
			7
			3 3 3 3 3 3 3
		}{
			7
		}%
	\end{example}
	
	\Note
	\begin{itemize}
		\item 对于$30\%$的数据,$1 \leq n \leq 5000$.
		\item 对于$100\%$的数据,$1 \leq n \leq 10^5$, $1 \leq a_i \leq 10^5$.
	\end{itemize}
	
\end{problem}

\begin{problem}{bamboo}{bamboo.in}{bamboo.out}{1 second}{256 MB}
    
    我们来种竹子吧.
    
    现在有$n$棵竹子被种下了,为了美观,我们希望最后竹子的高度分别为$a_1,a_2,\dots,a_n$.
    
    现在已知,每天竹子都会长高1cm,如果竹子被砍过一次,那么它就不会再生长.
    
    我们在种下竹子后,每$d$天去看它们一次,如果发现某棵竹子高度够了(当前高度大于等于我们希望的高度),我们就将它多余的部分砍掉(如果刚好合适,我们也算作砍过),我们希望我们砍下来的竹节的长度小于等于$k$,请问我们的$d$最大可以取多大?

    \InputFile

    第$1$行包含$2$个整数$n$和$k$;
    
    第$2$行包含$n$个整数$a_1,a_2,\dots,a_n$.

    \OutputFile

	输出一行,包含最大的整数$d$.

    \Example

    \begin{example}
        \exmp{
            3 4
            1 3 5
        }{
            3
        }%
    \end{example}

	样例解释: 第$3$天的时候,三个竹子的高度分别为$3, 3, 3$,我们发现前两棵够了,分别砍下$2, 0$长度的竹节,第$6$天的时候,第三棵竹子长到了$6$的高度,我们砍下$1$长度的竹节,最终砍下了$3$长度的竹节.
	
	\begin{example}
		\exmp{
			3 40
			10 30 50
		}{
			32
		}%
	\end{example}

    \Note
    
    \begin{itemize}
    	\item 对于$30\%$的数据,$1 \leq a_i \leq 100$, $1 \leq k \leq 10000$;
        \item 对于$100\%$的数据,$1 \leq n \leq 100$, $1 \leq a_i \leq 10^9$, $1 \leq k \leq 10^{11}$,保证每个点不相同.
    \end{itemize}

\end{problem}



\end{document}
