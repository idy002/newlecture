\documentclass[11pt,serif]{beamer}
\usepackage[utf8]{inputenc}
\usepackage[T1]{fontenc}
\usepackage{xeCJK}
\usepackage{amsmath}
\usepackage{amsfonts}
\usepackage{amssymb}
\usepackage{graphicx}
\usetheme{Boadilla}
\begin{document}
	\author{丁尧尧}
	\title{Solution}
	\subtitle{day1}
	\date{\today}
	
	\begin{frame}[plain]
		\maketitle
	\end{frame}

	\begin{frame}{shortway}
		\begin{solution}
			我们考虑总共有$k+1$个时,很显然菊花图是最优的.
			
			我们新加一个内部点时,我们可以加在中心点到某个叶子节点的边上.
			
			不断加点,我们让从中心点到各个叶子节点的距离尽量平均(相差不超过1), 直到我们有$n$个点.
			
			这样一定是最优的. 
			
			(假如一个图是最优解,我们看这个图的直径,然后选择这条直径最中间的点为中心点,然后我们必定可以将这个最优解调整成我们那种形式而答案不变劣).
		\end{solution}
	\end{frame}

	\begin{frame}{evolution} 
		\begin{solution}
			线段树维护. 每个节点维护,在当前节点对应区间里面的字符,模$mod$余$r$的$c$字符有多少个,这样需要$10 \times 11 \times$的空间.
		\end{solution}	
	\end{frame}

	\begin{frame}{mst}
		\begin{solution}
			我们首先求一个mst出来,然后将mst的树边和非树边分类讨论.
			
			
			对于一条树边$e_1$, 对于另外一条边$e_2$,如果生成树加上$e_2$后$e_1$在形成的环上,那么$e_1$能取的权值必须小于$e_2$的权值,我们求个最小值再减一就得到$e_1$的答案,如果不存在这样的$e_2$,答案为$-1$.
			
			
			对于一条非树边,假设$u$,$v$是它的两个端点,这条非树边的答案就是树上$u$,$v$之间的最大边权减一.
			
			
			以上所有信息都可以用树链剖分维护,见代码.
		\end{solution}
	\end{frame} 

\end{document}