\documentclass[11pt,a4paper,oneside]{article}
\usepackage[english]{babel}
\usepackage{olymp}
\usepackage[dvips]{graphicx}
\usepackage{color}
\usepackage{colortbl}
%\usepackage{expdlist}
%\usepackage{mfpic}
%\usepackage{comment}
\usepackage{multirow}

\usepackage{xeCJK}

%\setCJKmainfont[BoldFont={Hei}]
%{SimSun}
%\setCJKmonofont{FangSong}

\renewcommand{\contestname}{
No.7 High School OI Training \\
idy002, \today
}    

\begin{document}

\begin{problem}{shortway}{shortway.in}{shortway.out}{1 second}{256MB}

    请你构造一个包含$n$个点,其中有正好$k$个叶子节点的无根树,请问这个树的直径\footnote{一棵无根树的直径是任意两个点之间的距离的最大值}最小是多少?

    \OutputFile

	输出一个数,表示最小直径.

    \Example

    \begin{example}
        \exmp{
            3 2
        }{
            2
        }%
    \end{example}

	样例解释:对应的边分别是:$(1,2), (2,3)$,对应直径为$2$.

    \begin{example}
		\exmp{
			5 3
		}{
			3
		}%
	\end{example}

	样例解释:一种可能的构造的边是:$(1,2), (2,3), (3,4), (3,5)$,对应直径为$3$.

    \Note
    
    \begin{itemize}
		\item 对于$30\%$的数据,$1 \leq n \leq 100$, $k \in \{ 2, 3, n - 1 \}$;
		\item 对于$100\%$的数据,$1 \leq n \leq 10^5$, $2 \leq k \leq n-1$.
    \end{itemize}

\end{problem}

\begin{problem}{evolution}{evolution.in}{evolution.out}{4 second}{512 MB}
    
    给你一个长为$n$的字符串$s$,包含AGCT四种字符,现在有两种操作:
    
    \begin{itemize}
    	\item 1 x c  表示将$x$这个位置的字符替换成$c$.
    	\item 2 l r e  表示一个询问,其中$l$,$r$是正整数, $e$是一个只包含AGCT的字符串,我们将$l$道$r$这段区间的字符串单独拿出来写下,然后在它下面写下$eee\dots$(重复无穷次),询问的答案就是这两个字符串字符相同的位置的个数.(询问不改变原字符串)
    \end{itemize}

    \InputFile

    第$1$行包含$1$个字符串$s$.
    
    第$2$行包含一个整数$q$表示操作数.
    
    接下来$q$行每行表示一个操作.格式见题目描述.

    \OutputFile

	对于每个询问,输出一行,包含一个整数表示答案.

    \Example

    \begin{example}
        \exmp{
            ATGCATGC
            4
            2 1 8 ATGC
            2 2 6 TTT
            1 4 T
            2 2 6 TA
        }{
            8
            2
            4
        }%
    \end{example}

	样例解释:对于第一个询问,所有对应字符都相同,故答案为$8$;对于第二个询问,我们提取的字符串是\verb|TGCAT|,我们对照的字符串是\verb|TTTTTT...|,相同的位置有$2$个;对于第三个询问,在字符串改变后,我们提取的字符串是\verb|TGTAT|,对照的字符串是\verb|TATAT...|, 有$4$个位置匹配.
	
	\begin{example}
		\exmp{
			GAGTTGTTAA
			6
			2 3 4 TATGGTG
			1 1 T
			1 6 G
			2 5 9 AGTAATA
			1 10 G
			2 2 6 TTGT
		}{
			0
			3
			1
		}%
	\end{example}

    \Note
    
    \begin{itemize}
    	\item 对于$30\%$的数据,$1 \leq \vert s \vert \leq 10^3$, $1 \leq q \leq 10^3$;
        \item 对于$100\%$的数据,$1 \leq \vert s \vert \leq 10^5$, $1 \leq q \leq 10^5$, $1 \leq l \leq r \leq \vert s \vert$, $1 \leq x \leq \vert s \vert$, $1 \leq \vert e \vert  \leq 10$, $c \in \{A, T, C, T\}$.
    \end{itemize}

\end{problem}

\begin{problem}{mst}{mst.in}{mst.out}{2 second}{256 MB}

    给你一个$n$个点,$m$条边的无向图,无重边无自环,边带边权.
    
    对于每条边,你需要求一个最大的权值$c$,使得当这条边的权值变成$c$后,这条边存在于任意最小生成树中,如果$c$可以任意大,输出$-1$.
    
    你需要对每条边输出答案.
    
    \InputFile

    第$1$行包含两个整数$n$, $m$表示图的点数和边数;
    
    接下来$m$行,每行包含三个整数$u, v, w$表示一条连接$u$, $v$的边,边权为$w$.

    \OutputFile
    
    输出一行,包含$m$个整数, 按输入中边的给出顺序输出对应答案.

    \Example

    \begin{example}
        \exmp{
            4 4
            1 2 2
            2 3 2
            3 4 2
            4 1 3
        }{
            2 2 2 1 
        }%
    \end{example}

	\begin{example}
		\exmp{
			4 3
			1 2 2
			2 3 2
			3 4 2
		}{
			-1 -1 -1 
		}%
	\end{example}

    \Note
    \begin{itemize}
        \item 对于$30\%$的数据,$1 \leq n, m, w \leq 500$.
        \item 对于$100\%$的数据,$1 \leq n, m \leq 10^5$, $1 \leq u, v \leq n$, $u \neq v$, $1 \leq w \leq 10^9$.
    \end{itemize}

\end{problem}


\end{document}
