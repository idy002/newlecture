\documentclass[11pt,a4paper,oneside]{article}
\usepackage[english]{babel}
\usepackage{olymp}
\usepackage[dvips]{graphicx}
\usepackage{color}
\usepackage{colortbl}
%\usepackage{expdlist}
%\usepackage{mfpic}
%\usepackage{comment}
\usepackage{multirow}

\usepackage{xeCJK}

%\setCJKmainfont[BoldFont={Hei}]
%{SimSun}
%\setCJKmonofont{FangSong}

\renewcommand{\contestname}{
No.7 High School OI Training \\
idy002, \today
}    

\begin{document}

\begin{problem}{recover}{recover.in}{recover.out}{1 second}{256MB}

    红红在恢复一个字符串.
    
    现在,红红得到了$n$个资料,第$i$个资料显示,在$x_{i,1}, x_{i, 2}, \dots, x_{i, k_i}$这些位置开始,都有同一个子串$s_i$.
    
    现在红红请你帮忙根据已有资料,恢复这个字符串,如果有多个可能的字符串满足条件,输出字典序最小的那个字符串.
    
    \InputFile

	第$1$行包含一个整数$n$;
	
	接下来$n$行,每行包含一个资料$s_i \; k_i \; x_{i,1} \; x_{i,2} \; \dots \; x_{i, k_i}$.

    \OutputFile

	输出要求的字符串.

    \Example

    \begin{example}
        \exmp{
            3
            a 4 1 3 5 7
            ab 2 1 5
            ca 1 4
        }{
            abacaba
        }%
    \end{example}

    \begin{example}
		\exmp{
			1
			a 1 3
		}{
			aaa
		}%
	\end{example}

    \begin{example}
		\exmp{
			3
			ab 1 1
			aba 1 3
			ab 2 3 5
		}{
			ababab
		}%
	\end{example}

    \Note
    
    \begin{itemize}
		\item 对于$30\%$的数据,$1 \leq q \leq 100$, $1 \leq n_i \leq 100$;
		\item 对于$100\%$的数据,$1 \leq n \leq 10^5$, $1 \leq n_i \leq 10^9$.
    \end{itemize}

\end{problem}

\begin{problem}{cross}{cross.in}{cross.out}{1 second}{256 MB}
    
    给你$n$个点:$(x_1,y_1),(x_2,y_2),\dots,(x_n,y_n)$.每个点你有三种可能的操作:
    \begin{itemize}
    	\item 画一条水平线穿过该点
    	\item 画一条竖线穿过该点
    	\item 什么都不做
    \end{itemize}

	多条直线重叠算一条直线,请问你可能画出多少种图形?

    \InputFile

    第一行一个整数$n$表示点数;
    
    接下来$n$行,每行两个整数$x_i,y_i$表示一个点.

    \OutputFile

	可能的图形个数,模$10^9+7$.

    \Example

    \begin{example}
        \exmp{
            2
            -1 -1
            0 1
        }{
            9
        }%
    \end{example}
	样例解释:两个点都有三种选择,总共有$3^2 = 9$种可能图形.
	
	\begin{example}
		\exmp{
			4
			1 1
			1 2
			2 1
			2 2
		}{
			16
		}%
	\end{example}
	样例解释:最多可以画出4条线,且这四条线的任意子集都可以画出,所有有$2^4 = 16$种可能图形.

    \Note
    
    \begin{itemize}
    	\item 对于$30\%$的数据,$1 \leq n \leq 12$;
        \item 对于$100\%$的数据,$1 \leq n \leq 10^5$, $-10^9 \leq x_i, y_i \leq 10^9$,保证每个点不相同.
    \end{itemize}

\end{problem}

\begin{problem}{string}{string.in}{string.out}{1 second}{256 MB}

    给你$n$个01字符串:$s_1,s_2,\dots,s_n$,有$m$个操作,第$i$个操作将$s_{a_i}$和$s_{b_i}$拼接起来形成一个新的字符串,记为$s_{n+i}$.每个操作之后,你需要输出一个最大的$k$,使得所有长度为$k$的01串(有$2^k$个)都是$s_{n+i}$的子串,如果不存在这样的$k$,则输出$0$.
    
    \InputFile

    第$1$行包含一个整数$n$;
    
    接下来$n$行每行包含一个01字符串;
    
    接下来$1$行包含一个整数$m$;
    
    接下来$m$行每行包含两个整数$a_i, b_i$表示一个操作.

    \OutputFile
    
    对于每个操作,输出一行,包含一个整数,表示对应操作的答案.

    \Example

    \begin{example}
        \exmp{
            5
            01
            10
            101
            11111
            0
            3
            1 2
            6 5
            4 4
        }{
            1
            2
            0
        }%
    \end{example}

	样例说明:第一个操作得到的字符串是\verb|0110|,包含\verb|0|和\verb|1|,且不包含\verb|00|,故答案为1.
	
	第二个操作得到的字符串是\verb|01100|,包含所有长度为2的01串,答案为2.
	
	第三个操作得到的字符串是\verb|1111111111|,不包含\verb|0|,故答案为0.

    \Note
    \begin{itemize}
        \item 对于$30\%$的数据,$1 \leq n, m \leq 10$.
        \item 对于$100\%$的数据,$1 \leq n, m \leq 100$, $1 \leq \vert s_i \vert \leq 100$ for all $1 \leq i \leq n$, $1 \leq a_i, b_i \leq n+i-1$.
    \end{itemize}

\end{problem}


\end{document}
