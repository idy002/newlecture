\documentclass[11pt,serif]{beamer}
\usepackage[utf8]{inputenc}
\usepackage{xeCJK}
\usepackage[T1]{fontenc}
\usepackage{amsmath}
\usepackage{amsfonts}
\usepackage{amssymb}
\usepackage{bbm}
\usepackage{graphicx}
\usetheme{Boadilla}

\begin{document}
	\author{丁尧尧}
	\title{题目选讲}
	\date{\today}
	
	\begin{frame}[plain]
	\maketitle
	\end{frame} 

	\begin{frame}{百度之星2018复赛 T1}
		
		\begin{problem}{没有兄弟的舞会}
			给你一棵$n$个点有根树,每个点有个权值,如果两个点有相同的父亲,则他们是兄弟。你现在需要找一个点集,其中最多有一对点是兄弟关系,问这个点集的权值和是多少?
			
			$1 \leq n \leq 10^5$.
		\end{problem}
		\pause
		\begin{solution}
			贪心或树型DP
		\end{solution}
	\end{frame}

	\begin{frame}{百度之星2018复赛 T2}
		\begin{problem}[序列期望]
			令$X_1,X_2,\dots,X_n$是整数随机变量,其中$X_i$是从$[l_i,r_i]$中随机选择的一个整数,令
			\begin{align}
				h & = max(X_1,X_2,\dots,X_n) \\
				Y & = \prod_{i = 1}^{n}(h + 1 - X_i)
			\end{align}
			问模$10^9+7$意义下$\mathbb{E}[Y]$.
			
			$1 \leq n \leq 100, 1 \leq l_i \leq r_i \leq 10^4$.
		\end{problem}
		\pause
		\begin{solution}
			枚举$h$
		\end{solution}
	\end{frame}

	\begin{frame}{百度之星2018复赛 T3}
		\begin{problem}[带劲的and和]
			给你一个$n$个点$m$条边的无向图,每个点有一个非负权值,点$i$的权值用$v_i$表示,令
			$$f(i,j) = \mathbbm{1}_{\text{$i$和$j$连通}}$$求:
			$$
				\sum_{i=1}^{n-1} \sum_{j=i+1}^{n} f(i,j) \times \max(v_i, v_j)\times (v_i \& v_j)
			$$
			
			$1 \leq n \leq 10^5$
		\end{problem}
		
		\pause 
	
		\begin{solution}
			先求连通块,对于同一个连通块的所有权值从小到达排序,然后从前往后算,其中$v_i \& v_j$需要记录每一位出现多少次。
		\end{solution}
	\end{frame}

	\begin{frame}{2018 Multi-University Training Contest 8}
		
		\begin{problem}[Character Encoding]
			求
			$$
				\sum_{i = 1}^{m}x_i = k \quad (0 \leq x_i < n)
			$$
			的方案数(模$10^9+7$)。
			
			$1 \leq m, k, n \leq 10^5$.
		\end{problem}
	
		\pause
		
		\begin{solution}
			容斥
		\end{solution}
	
	\end{frame}

	\begin{frame}{2018 Multi-University Training Contest 8}
		\begin{problem}[Card Game]
			给定$n$张卡片,每张卡片正反面各有一个数。问至少要翻转多少张卡片,才能使正面向上的数互不相同,并求方案数。
		\end{problem}
		
		\pause 
	
		\begin{solution}
			首先建图:每个数字为一个节点,每张卡片反面数字向正面数字连一条有向边。问题转化为:至少要反转多少条边的方向,才能使得每个点的入度不会超过1。我们对每个弱连通分量分别处理。易知,当底图是树或基环树时,才可能有解。对于基环树,先把环找出来,然后将环上的边的方向统一一下;非环边的方向则是唯一确定的,从环上的点向外做一遍dfs 即可。对于树,可以正反两次dfs处理出每个点作为根时所需要的反向次数,并统计出最小值以及方案数。最后将答案合并即可。
		\end{solution}
	\end{frame} 

	\begin{frame}{2018 Multi-University Training Contest 8} 
		\begin{problem}[Taotao Picks Apples]
			对于一个序列,从前往后看,每次当手上没有数或者手上的数小于当前的数,就把手上的数替换成当前的数,定义手上出现的数的个数为这个序列的可见度。给定一个长为$n$的序列,有$m$个询问,每个询问两个数$(p,q)$,表示如果把$p$位置的数换成$q$,这个序列的可见度为多少?
			
			$1 \leq n, m \leq 10^5$
		\end{problem}
		\pause
		\begin{solution}
			法一:线段树维护
			
			\pause 
			
			法二:考虑每次修改不叠加,因此我们可以从如何对原序列进行预处理着手。
			通过观察可以发现,将原序列从任意位置断开,我们可以通过分别维护左右段的某些信息来拼接得到答案。	
			对于每次询问:
			考虑这个数左边的部分加上这个数之后的答案和最大值;
			再找到右边第一个大于左半部分最大值的数,答案相加即可。
		\end{solution}
	\end{frame} 

	\begin{frame}{2018 Multi-University Training Contest 8} 
		\begin{problem}[Pop the Balloons]
			给定一个$m\times n$气球矩阵,扎掉一个气球后,同行同列的气球都消失。问对于每个$1 \leq x \leq k$,扎恰好$x$次能够清除所有气球的方案数。
			
			$1 \leq n \leq 20, 1 \leq m \leq 12$.
		\end{problem}
	\end{frame}
	\begin{frame} 
		\begin{solution}
			显然扎掉的气球两两不同行且不同列。只需要求出扎x个气球的集合,然后乘以x!即可。
			
			由于行数较少,考虑枚举扎掉的行集合,设扎掉的行的bitmap为mask1。设dp[r][mask2]为考虑前r列,已经扎掉的气球所在行为mask2的方案数。考虑状态dp[r][mask2]的转移:
			\begin{itemize}
				\item 如果第i+1i+1列的气球被mask1包含,则dp[r+1][mask2] += dp[r][mask2];
				\item 对于第i+1i+1列的每个在mask1中,但不在mask2中的气球w,我们可以将它扎掉,即为dp[r+1][mask2|w] += dp[r][mask2];
			\end{itemize}
			总复杂度:$O(nm3^m)$
			
			亦可以把最开始枚举的mask1放到状态里和mask2合并,复杂度不变。
		\end{solution}
	\end{frame}

	\begin{frame}{A little tricky problem}
		\begin{problem}
			给你 $1 \leq x_1, x_2, \dots, x_k \leq n$, 请计算:
			$$
				gcd(2^{F(x_1)}-1, 2^{F(x_2)}-1, \cdots, 2^{F(x_n)}-1)
			$$
			其中$F(n)$是斐波那契数列:$F(0) = 0, F(1) = 1, F(2) = 1, \dots$.
		\end{problem}
		\pause
		\begin{solution}
			$$
				ans = 2^{F(gcd(x_1,x_2,\dots,x_n))}-1
			$$
		\end{solution}
	\end{frame}

	\begin{frame}{2018 Multi-University Training Contest 6}
		\begin{problem}[bookshelf]
			有 $N$ 本一摸一样的书,有一个共有$K$ 层的书架,现在要把书都放到书架上。
			放完后假设第 $i$ 层书架有 $s_i$本书,则该层书架的稳固值为 $2^{F(s_i)}-1$。
			定义整个书架的美观值为所有层书架的稳固值的GCD。
			问现在随机放这些书($N=s_1+s_2+\cdots+s_K$,两个放法不同当且仅当某个$i$使得$s_i \neq s_i'$),整个书架的美观值的期望值是多少。
			
			$1 \leq N, K \leq 10^5$,答案模$10^9+7$。
		\end{problem}
		\pause
		\begin{solution}
			我们本质要求$d = gcd(s_1,s_2,\dots,s_K)$的分布,枚举$d$,然后求gcd是$d$倍数的方案的个数,然后容斥。
		\end{solution}
	\end{frame}

	\begin{frame}{2018 Multi-University Training Contest 6}
		\begin{problem}[Shoot Game]
			平面上,有$n$个障碍物,每个障碍物由$(H,L,R,w)$描述,表示障碍物是一条在高度是$H$,x坐标从$L$到$R$的闭线段,强度为$w$。
			
			你现在可以从原点发射一些射线,每条射线的能量由你定,发射后如果碰到强度小于等于能量的障碍物,则障碍物被清除,射线继续向前,且能量不变;否则射线消失。
			
			请问最少需要发射多少能量的射线才能清除所有障碍物?
			
			$1 \leq n \leq 300$, $1 \leq H \leq 10^9$, $-10^9 \leq L, R \leq 10^9$, $0 \leq w \leq 10^9$.
		\end{problem}
	\end{frame}
	\begin{frame}
		\begin{solution}
			任何一种方案都可以转化成射向端点的方案,然后我们就考虑向这$2n$个端点发射射线。
			
			然后用$dp[l][r]$表示,消灭所有完全包含在$l$到$r$这段的线段最少需要多少能量,因为每个区间中,能量最大的一定会被消灭,只需要枚举我们消灭能量最大的障碍在哪个节点就行。
		\end{solution}
	\end{frame}

	\begin{frame}
		\begin{problem}
			先来道水题轻松一下.
			
			对于一个字符串,我们可以有两种操作:
			\begin{itemize}
				\item 在最后加B.
				\item 将字符串翻转后,在最后加上A.
			\end{itemize}
			现在给出两个串$A,B$,请你求出最长的一个串$C$,使得$C$可以通过不断做以上两种操作变成$A$和$B$,输出$C$.
		\end{problem} 
	\end{frame}

	\begin{frame}{bzoj1776}
		\begin{problem}
			一$n$个点的棵树,每个点有一个颜色,请问对于每种颜色而言,距离最远的一对点的距离是多少?
		\end{problem}	
	
		\pause 
		
		\begin{solution}
			一定可以选该颜色最深的点作为一个端点.
		\end{solution}
		
	\end{frame}

	\begin{frame}
		\begin{problem}
			请你维护一个向量集合,支持以下操作:
			\begin{itemize}
				\item 加入一个向量(保证现在没有该向量)
				\item 删除一个向量(保证以前加入过)
				\item 询问一个向量是否可以被当前向量集合线性表示
			\end{itemize}
		\end{problem} 
		
		\pause
		
		\begin{solution}
			时间线段树
		\end{solution} 
	\end{frame}

	\begin{frame}
		\begin{problem}
			一个$n\times m$的网格,上面有些地方是海,有些地方是陆地,已知陆地是四连通的,请问最少删掉几个陆地,使得陆地不再四连通?
		\end{problem}
		
		\pause
		
		\begin{solution}
			hint1: 最多删除两个点
			
			\pause
			
			hint2: 双联通
		\end{solution} 
	 
	\end{frame}

	\begin{frame}{随机游走}
		\begin{problem}
			给你一棵$n$个点的树,从$1$号节点开始,每次随机选择一条边走过去,请问走到$n$号点的期望时间是多少.
		\end{problem}
	
	
		\pause
		
		\begin{solution}
			期望的线性性.
		\end{solution} 
		 
	\end{frame}



	\begin{frame}{CF 371 Div1 C}
		\begin{problem}
			给你$n$个整数$a_1,a_2,\dots,a_n$,你每次操作可以选择一个数,然后将这个数加1或减1,请问你最少需要操作多少次,使得该序列不降?
			
			$1 \leq n \leq 3000$, $1 \leq a_i \leq 10^9$.
			
			\pause
			
			提示: 一定存在一种最小方案,使得最终序列的每个数在一开始出现过.
			
			\pause
			
			如果将问题改成"使得该序列严格递增",应该怎么做?
		\end{problem}
	\end{frame}

	
	
\end{document}