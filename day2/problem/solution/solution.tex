\documentclass[11pt,serif]{beamer}
\usepackage[utf8]{inputenc}
\usepackage[T1]{fontenc}
\usepackage{xeCJK}
\usepackage{amsmath}
\usepackage{amsfonts}
\usepackage{amssymb}
\usepackage{graphicx}
\usetheme{Boadilla}
\begin{document}
	\author{丁尧尧}
	\title{Solution}
	\subtitle{day1}
	\date{\today}
	
	\begin{frame}[plain]
		\maketitle
	\end{frame}

	\begin{frame}{split}
		\begin{solution}
			$4$是最小的合数,然后我们贪心地取,最后根据$n \bmod 4$来特判一下。
		\end{solution}
	\end{frame}

	\begin{frame}{cross} 
		\begin{solution}
			建一个图,如果一个两个点同$x$坐标或同$y$坐标,则把它们连边,分别求解每个连通块的答案,再乘起来.
			
			对于一个连通块,如果它是一棵树,则答案为$2^{x+y} - 1$,否则为$2^{x+y}$,其中$x$是所有点$x$坐标集合的大小,$y$同理.
		\end{solution}	
	\end{frame}

	\begin{frame}{string}
		\begin{solution}
			考虑一个长度为$k$的所有01串有$2^k$个,而一开始我们所有长度为$k$的01串最多有$100k$个,每次合并最多增加$k$个,合并$100$次增加$100k$,最终有$2^k \leq 200k$,解得$k \leq 12$.
			
			所以我们只需要考虑长度小于等于$12$的01串即可,然后就各种操作都可以啦.
		\end{solution}
	\end{frame} 

\end{document}