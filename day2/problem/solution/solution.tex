\documentclass[11pt,serif]{beamer}
\usepackage[utf8]{inputenc}
\usepackage[T1]{fontenc}
\usepackage{xeCJK}
\usepackage{amsmath}
\usepackage{amsfonts}
\usepackage{amssymb}
\usepackage{graphicx}
\usetheme{Boadilla}
\begin{document}
	\author{丁尧尧}
	\title{Solution}
	\subtitle{day1}
	\date{\today}
	
	\begin{frame}[plain]
		\maketitle
	\end{frame}

	\begin{frame}{prob}
		\begin{solution}
			如果可以选择一个题集满足条件,那么其中会的人最少的题最多有k/2个人会.我claim这些人在这个题集中一定存在一道题必定所有人都没做过(当k<=5). 所以我们只需要看是否存在两道题满足条件即可.
		\end{solution}
	\end{frame}

	\begin{frame}{pizza} 
		\begin{solution}
			我们可以通过
			$$
				P = \left \lceil \frac{\sum_{i = 1}^{N}s_i}{S} \right \rceil
			$$
			算出我们需要买的披萨个数.我们接下来需要判断两种类型的披萨的分配.
			
			假设我们买$a$个第一种披萨,$P-a$个第二种披萨. 
			
			我们可以根据$a-b$将人从排序,那么我们必定先让前面的人吃第一种披萨,后面的人吃第二种披萨. 然后由正变负的地方就是决策点,那个披萨可能是第一种,也可能是第二种,较大值就是答案.
		\end{solution}	
	\end{frame}

	\begin{frame}{scream}
		\begin{solution}
			我们比较容易想到的一个dp是用$dp[i][j]$前$i$个数,分成$j$段,最大的美味值. 转移为:
			$$
				dp[i][j] = \max_{k = 1}^{i-1}dp[k-1][j-1] + c(k,i)
			$$
			然后我们发现$dp[\cdot][j]$仅由$dp[\cdot][j-1]$转移过来,我们把$j$压掉,然后用数据结构(线段树)维护右边的max值.
		\end{solution}
	\end{frame} 

\end{document}