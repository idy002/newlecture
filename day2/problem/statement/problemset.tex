\documentclass[11pt,a4paper,oneside]{article}
\usepackage[english]{babel}
\usepackage{olymp}
\usepackage[dvips]{graphicx}
\usepackage{color}
\usepackage{colortbl}
%\usepackage{expdlist}
%\usepackage{mfpic}
%\usepackage{comment}
\usepackage{multirow}

\usepackage{xeCJK}

%\setCJKmainfont[BoldFont={Hei}]
%{SimSun}
%\setCJKmonofont{FangSong}

\renewcommand{\contestname}{
No.7 High School OI Training \\
idy002, \today
}    

\begin{document}

\begin{problem}{prob}{prob.in}{prob.out}{1 second}{256MB}

	有$n$道题目,$m$个有经验的队伍,每个队伍可能做过一些题目.
	
	现在,请你判断,我们是否可以选出一些题目,使得在这些题目中,对于每个队伍,做过的题目小于等于没做过的题目.

    \InputFile

	第$1$行$1$个整数$T$表示数据组数;
	
	对于每组数据:
	
	第$1$行包含两个整数$n$和$m$,分别表示题目数和队伍数;
	
	接下来$n$行,每行包含$m$个数字($0$和$1$),表示该题是否被某个队伍做过(第$i$个数字表示是否被第$i$个队伍做过).

    \OutputFile

	对于每组数据,输出"YES"或者"NO",表示是否能找到题目要求的题目集合.

    \Example

    \begin{example}
		\exmp{
			2
			5 3
			1 0 1
			1 1 0
			1 0 0
			1 0 0
			1 0 0
			3 2
			1 0
			1 1
			0 1
			
		}{
			NO
			YES
		}%
	\end{example}

	样例解释:
	
	第一组数据:因为第一个队伍做过所有题目,所以条件不可能达成.
	
	第二组数据:可以选择第一道和第三道题目,这样每个队伍做过的题目和没做的题目都是$1$道,所以可以.

    \Note
    
    \begin{itemize}
		\item 对于$30\%$的数据,$1 \leq m \leq 2$;
		\item 对于$100\%$的数据,$1 \leq m \leq 4$, $1 \leq n \leq 10^5$, $1 \leq T \leq 100$, 所有组数据的$n$的和小于等于$2 \times 10^5$..
    \end{itemize}

\end{problem}

\begin{problem}{pizza}{pizza.in}{pizza.out}{1 second}{256 MB}
    
    JJ要买披萨.
    
    有两种披萨,且每种都有无限份,JJ可以各买一些份数,每份都要切成$S$片.
    
    现在JJ家里来了$N$位客人,第$i$位客人要吃$s_i$片披萨才能吃饱,吃一片第一种披萨会有$a_i$的快乐值,吃一片第二种披萨会有$b_i$的快乐值.
    
    现在JJ要定披萨,JJ不希望浪费,所以JJ在让每位客人都能吃饱的前提下,会买尽量少的披萨.
    
    请问所有客人的快乐值之和最大能多少?

    \InputFile

    第$1$行$2$个整数$N$,$S$分别表示客人的数量和一份披萨能分成多少片.
    
    接下来$N$行,第$i$行三个数$s_i, a_i, b_i$.

    \OutputFile

	输出一个数,表示答案.

    \Example

    \begin{example}
        \exmp{
            3 12
            3 5 7
            4 6 7
            5 9 5
        }{
            84
        }%
    \end{example}

	样例解释:只需要买一个披萨就够了,如果买第一种披萨,总的快乐值是$5 \times 3 + 6 \times 4 + 9 \times 5 = 84$,买第二种披萨对应的快乐值是$7 \times  3 + 7 \times 5 + 5 \times 5 = 81$,所以只需要买一个第一种类的披萨,可以获得84的快乐值.
	
	\begin{example}
		\exmp{
			6 10
			7 4 7
			5 8 8
			12 5 8
			6 11 6
			3 3 7
			5 9 6
		}{
			314
		}%
	\end{example}

    \Note
    
    \begin{itemize}
    	\item 对于$30\%$的数据,$1 \leq N \leq 1000, 1 \leq s_i \leq 10$;
        \item 对于$100\%$的数据,$1 \leq N, S, s_i, a_i, b_i \leq 10^5$.
    \end{itemize}

\end{problem}

\begin{problem}{scream}{scream.in}{scream.out}{4 second}{256 MB}

    你面前按顺序排列着$n$块冰淇淋(种类可能不同),你手中有$m$个冰淇淋桶,你现在需要将这$n$块冰淇淋分成$m$份,要求每份冰淇淋是连续的一段,然后将这$m$份冰淇淋分别装入$m$个冰淇淋桶中.
    
    第$i$个冰淇淋桶的美味值是装入其中的冰淇淋的种类数,请问怎样分冰淇淋才能让美味值之和尽量大?
    
    \InputFile

    第$1$行包含$2$个整数$n$,$m$;
    
    第$2$行包含$n$个整数数$a_1,a_2,\dots,a_n$表示冰淇淋的种类.

    \OutputFile
    
	输出一个整数表示美味值之和.

    \Example

    \begin{example}
        \exmp{
            4 1
            1 2 2 1
        }{
            2
        }%
    \end{example}

	样例说明:只能将所有的冰淇淋装入一个桶中,因为有两种冰淇淋,所以美味值位$2$.
	
	\begin{example}
		\exmp{
			7 2
			1 3 3 1 4 4 4
		}{
			5
		}%
	\end{example}
	
	样例说明:最优分法是$\{1, 3\}, \{3, 1, 4, 4, 4\}$,美味值分别为$2, 3$,和为$5$.
	
	\begin{example}
		\exmp{
			8 3
			7 7 8 7 7 8 1 7
		}{
			6
		}%
	\end{example}

    \Note
    \begin{itemize}
        \item 对于$30\%$的数据, $1 \leq n \leq 1000$;
        \item 对于$100\%$的数据,$1 \leq n \leq 20000$, $1 \leq a_i \leq n$, $1 \leq m \leq min(50,n)$.
    \end{itemize}

\end{problem}


\end{document}
